\documentclass[a4paper, 10pt]{jsarticle}
\usepackage{templates/ikenote}  


%\usepackage{jumoline} % 日本語改行対応下線など
%\usepackage{ulem}
\usepackage{hyperref}

\newcommand{\amkk}[1]{\colorbox[rgb]{0.9, 0.9, 0.9}{#1}}
\newcommand{\colsnumber}[1]{#1}
% 表の行間
\renewcommand{\arraystretch}{0.9}


% 行番号を表示する
\usepackage[right]{lineno}
\pagewiselinenumbers  
\modulolinenumbers[5]

% ヘッダー用の変数設定 
\newcommand{\LectureName}{修論指導会 研究計画発表ver.1.0}
\newcommand{\DocumentCategory}{}
\newcommand{\LectureDay}{\today}
\newcommand{\MyName}{池田 了哉 (指導教員 : 京免 徹雄)}
\newcommand{\MyTitle}{}

\newcommand{\MyReportTitle}{題目}
\newcommand{\MyReportSubTitle}{\\ 副題}

%\title{日本型高卒就職システムと離職の実態}
\title{\vspace{-15mm}\textbf{\Large \MyReportTitle {\large \MyReportSubTitle \\ \vspace{-3mm}  \MyTitle \MyName}}}

%\author{\MyTitle \MyName}

%%%%%%%%%%%%%%%%%%%%%%%%%%%%%%%%%%%%%%%%%%%%%%%%%%%%%%%%%%%%%%%%t%%%%%%%%%%%%%%%%%%%%%%%%%%%%%%%%%%%%%%%%%%%
\begin{document} % 本文 ここから ============================================================================

\maketitle
\firstpage

\vspace{-15mm}
%\begin{multicols}{2}

\section{問題の所在と研究の目的}

若年者の早期離職は、転職回数が多い人ほど年収が下がる傾向があること\citep[p.10]{work-report2024}などから、個人のキャリア形成に影響を及ぼすことが指摘され注目されてきた。
近年、新規学卒者の1/3以上が、就職後3年以内の「早期離職」を経験する。
「平成30年若年者雇用実態調査」によれば、離職経験を問わず若年者の6割以上が、「一つの会社に長く勤める」ことを望ましいと考えている。
このことから、離職経験者は、望ましいと考えているコースに反して、不本意に早期の離職を選んでいることが推測される。

同調査では、離職の理由も調べている。その中でも、「労働時間・休日・休暇の条件が良くなかった(35.2\%)」「仕事が自分に合わない(23.9\%)」「賃金の条件が良くなかった24.2\%」が挙げられていることに着目する。
これらの離職理由は、初職選択の時点で、情報を得ていれば、回避できるものも含まれている。
企業が求人票に記載した労働条件が現実のそれが大きく乖離していることも考えられるが、生徒の企業理解が不足している可能性もある。

高等学校に注目すると、自己理解を促すようなキャリア教育が、実施されていないことはない。
「キャリア教育に関する総合二次調査」\citep{Career-second-report}において、「自分の興味や関心、長所や短所などについて把握し、自分らしさを発揮すること」を、指導していると回答した高校教員(87.8\%)と、そのような指導を役立ったと回答している生徒(78.5\%)の割合は高い。
一方で、それを「自分の将来の生き方や進路について考えるため、指導してほしかったこと」として挙げる生徒も存在する。そして、このように答えた生徒の割合は、前回調査から増えている。
現在の実践されているキャリア教育は、生徒や離職を経験した者のニーズに適ったものではない可能性がある。

また、労働条件の悪い企業へ生徒・学生が就職することを防ぐ機能は存在している。
\citet{oshima-2012}は、大学就職部に注目し、条件の悪い求人を学生に紹介しないことで、条件の悪い企業へ学生が流れることを防ぐ、「セーフティーネット」としての機能があることを明らかにしている。
職業斡旋を学校が担う、高卒就職でも同様の役割が果たされていると考えることができる。
しかし、企業が虚偽の条件に基づいて求人した場合、それを見破らない限り、その機能は正しく作動しない。
この他にも、様々な外因で企業を取り巻く状況が変化した場合には、労働者本人の判断によって対応しなければならないため、生徒・学生には職業的に自立し、企業を選択する必要がある。


これら意味で、現在のキャリア教育は、早期離職問題に対応しきれていない可能性がある。

\citet{katayama-2016}は、工業高校における高卒就職の選考プロセスの前段階において、教師による「調整」が行われていることを描き出した。
高卒就職では、「一人一社制」の慣習が残っており、一人の生徒が志願できる企業の数には、求人数に基づき制限が設けられている。
ある企業を志望する生徒が重なった場合、教師は学業成績の低い生徒に対して、代替の企業を紹介し、志望者数を「調整」するのである。
また、\citet{nagasu-mimitsuka-repo-2003}は、教員へのインタビューの中で、企業と交渉し、求人数の方を「調整」している語りを引き出している。
片山の描く「調整」では、重複した生徒の希望を学業成績をもとに調整しているが、長須のインタビューでは、「基準はないです。総合的に。」と語られる。
「調整」の結果、「仕事が自分に合わない」企業へ振り向けられる可能性がある中で、調整の基準がどのようなものであるか、内部過程を教師の側から明らかにすることが必要である。


本研究では、高校のキャリア教育に着目する。
そして、「不本意早期離職」がキャリア教育・就職指導プロセスに内在する課題が顕在化したものであると捉え、その課題を明らかにすることを目的とする。

\section{研究課題と研究の方法}
キャリア教育に内在する課題を明らかにするために、以下の研究課題を設定する。

\noindent
\fbox{研究課題1} 進路指導部にある「セーフティーネット」があるのに、「労働条件が悪い」企業に就職する人がいるのはなぜか。

進路指導担当経験のある高校教師のへのインタビューを通じて、どのように求人票を扱われているかを明らかにする。
特に、どのような条件が「労働条件の悪さ」として判断し、「セーフティ—ネット」としての働きを作動させているか、また判断の材料となる情報はどのような媒体から得ているかを検討する。
勤務実態が、求人票に記載された内容と異なる可能性がある中で、どのような過程で教師は記載内容を信用するのかその過程を明らかにする。

%そして、この機能が普遍的なものかを明らかにする。

\noindent
\fbox{研究課題2} 自己理解を促す実践を先生方は意識しているのに、「仕事が自分に合わない」で辞める人がいるのはなぜか。 
また、この他の基礎的汎用的能力は育成されているのか。

自己理解を促す指導を評価する生徒と、実施してほしいと希望する生徒が共存している現状がなぜ起きているかを検討する。
高校への参与観察を通じて、具体的にどのような実践が行われているか、その一例を明らかにする。
そして、自己理解に関する指導のみならず、キャリア教育を通して育成される力である、「基礎的・汎用的能力」がどのように育成されているかを検討する。
また、その指導を受けた生徒がどのような成長を自覚するか、生徒へのインタビューから明らかにする。



\noindent
\fbox{研究課題3} 教師による「調整」はどのような基準で行われているか。また、「調整」はどのような原理で正当化されているか。

先行研究では、学業成績をもとに「調整」されている事例と、「総合的な判断」をもとに調整されている例があった。
このような違いがなぜ生じているのか、教師へのインタビューから考察する。
教師は、調整の結果生徒が希望を変更するように求めることをどのように正当化しているのか、をインタビューを中心に
明らかにする。


\section{対象事例の選定に向けて}
本研究では、高校の就職指導を取り上げることを計画している。
高校を取り上げるのは、高校等への進学率が非常に高く、ほぼすべての子どもが経由する教育機関であるためである。
中でも、就職者と進学者が混在する学校をフィールドとすることで、それぞれに対してどのような指導・価値づけが行われているかに迫る。
高校就職指導では、地域の産業に大きく影響される部分があるため、注目する地域の産業構造や、就職率を十分に考慮する。


\bibliography{ikeda01} %hoge.bibから拡張子を外した名前
\bibliographystyle{templates/jecon} %参考文献出力スタイル


%\end{multicols}
\end{document}

