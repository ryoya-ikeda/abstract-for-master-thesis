\subsection{私の問題関心}
私の問題意識の出発は、自身が高校生の頃に体験した、進路指導とそれが招いた結果に対する疑問です。

私は、過半数が卒業後就職する、工業高校の出身です。
そこでは、進路を選ぶ際に、成績順に受験する(推薦してもらう)企業や学校を選択するという仕組みでした。
(「好きな会社や大学に行きたかったら、いい成績を取れ」という指導を受けてきました。)

この制度のもと、学科(クラス)首席の友人は、卒業後に待遇のいい会社に就職しました。しかし、彼は、予想以上の激務に堪えきれず数ヶ月で退職し、それまで学んでいた建築とは異なる職に転職しました。
彼の心の裡を伺い知ることはできませんが、少なくとも首席を取るだけの努力はしていたはずです。その彼が、この制度の中で、それを投げ出す転職を選ばざるを得なかったことがずっと気になっています。

このことから、どうすれば彼のような離職を減らせるか、ということを考えています。

\subsection{用語の確認}
\begin{description}
	\item[職業指導] 仕事の世界に入る生徒に職業的発達(Vocational Development)を円滑にするため、進路発達(Career Development)を促進する過程であり、各人が自分の諸特性を把握、伸張し、
	職業に\ruby{就}{(ママ)}いての探索を深め、主体的に自分の進路を決定し、その後の職業生活を通じて自己表現できるような能力、態度、価値観を育てる活動\citep[Pp.102--103]{nonc-ito-2009}。
	\item[キャリア教育] 一人一人の社会的・職業的自立に向け、必要な基盤となる能力や態度を育成する。普通教育、専門教育を問わず様々な教育活動の中で実施され、職業教育も含む\citep[p.19]{mext2011-arikata}。
	\item[職業教育] 一定又は、特定の職業に従事するために必要な知識、技能、能力や態度を育成する。具体の職業に関する教育を通して行われる。この教育は、社会的・職業的自立に向けて必要な基盤となる能力や態度育成する上でも、極めて有効である\citep{mext2011-arikata}。
%	\item[メリトクラシー] 「人の評価は,出自・家柄などではなく,本人の知能・努力・業績によるべきだとする考え方。また,そのような考え方に立つ社会」(広辞苑(第 7 版))\\
%		\indent \ \ \ イギリスの社会学者ヤングによる造語。貴族社会(Aristovracy;家柄や門地になどを問うことで社会的な支配が確立していた時代)の対義語として用いられた\citep[p.110]{nonc-ito-2009}。
\end{description}
