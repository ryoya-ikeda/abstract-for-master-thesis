\section{大まかな研究計画}

修論締め切りまで残り : およそ555日\progressbar{0.86}
(2027/1/10を修論提出日と仮定した場合の日数)
%2025/4/1を起点とし、2027/1/10を修論提出日と仮定した場合の日数 : 649
%本日時点の残り日数 : 586
\begin{table}[h]
    \centering
    \caption{修士論文提出までの大まかな計画}
    \label{tab:hogehoge}
    \small
    \begin{tabular}{ccll}
        \hline
		年	& 時期	& \multicolumn{1}{c}{内容} 		& \multicolumn{1}{c}{備考}\\ \hline
		2025& 6月3日		& 発表\ajMaru{1}				& 済 \\
			& 7月8日		& 発表\ajMaru{2}				& 本日\\
			& 8月2〜4日 & 技術研究会@つくば 			&  \\
			&  9月		& 研究会(発表\ajMaru{3}) 		& \\
			& 10月		& キャリア教育学会 研究大会@大阪 & \\
			&  月 日		& 発表\ajMaru{4}	発表\ajMaru{5}			&  \\
		2026& 2月?		& 研究会(発表\ajMaru{6})		&	 \\
		\hline
		2026& 4月ごろ		& 発表\ajKuroMaru{1}			& 指導会要旨について\\
			& 5月下旬		&[第1回修士論文中間指導会]		& (2025/5/28)\\
			& 			& 発表\ajKuroMaru{2}			& 指導会を受けて修正点について\\
			&			& (キャリア教育学会発表申し込み締め切り)			& \\
			& 			& 研究会(発表\ajKuroMaru{3}) 	& 指導会要旨について\\
			& 9月中旬		&[修士論文仮題目 提出締切]		& (2025/9/16)\\
			& 9月下旬		&[第2回修士論文中間指導会]		& (2025/9/24)\\
			&  月 日		& 発表\ajKuroMaru{4}
						& →修論方向最終決定? \\
			&  月 日		& 発表\ajKuroMaru{5}			& 予備審査練習?\\
			& 12月下旬	& 修士論文・要旨 完成(目標)		& \\ \hdashline
		2027& 1月中旬		&[修士論文 提出締切]			& (2026/1/13)\\
			& 1月中旬		&[修士論文要旨 提出締切]		& (2026/1/14)	\\
			& 1月下旬		&[修士論文予備審査会] 			& (2026/1/21)\\
			& 			&研究会(発表\ajKuroMaru{6}) 	& (2026/1/21)\\
		\hline
    \end{tabular}
\end{table}

\begin{comment}
\section{修士論文の章構成}
\begin{chapenum}
  \item[\textbf{\hfill 序 章 \hfill}] あああ
    \begin{sectenum} 
      \item 背景
      \item 先行研究
    \end{sectenum}
  \item 
    \begin{sectenum}
      \item 理論的枠組み
    \end{sectenum}
\end{chapenum}
\end{comment}


